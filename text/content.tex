%% content.tex
%%

%% ==============
\chapter{Grundlagen}
\label{ch:Content1}
%% ==============

\section{Adaption}
Klärung des Begriffes Adaption. Was bedeutet Adaption auf die Software bezogen.


%% ===========================
\subsection{Adaption auf den Nutzer}
\label{ch:Content1:sec:Section1}
%% ===========================
Hinleitung der Adaption auf den Nutzer. Wie kommen wir zum Nutzer?
\dots

\section{Gamification}

\section{Problembasiertes Lernen}

\section{Recommendersysteme}


%% ===========================
\chapter{Nutzer}

\section{Nutzerrollen}
\subsection{Professoren}
\subsection{Studenten}
\subsection{Studierendensekretariat}
\subsection{Admins}

\section{Nutzer Klassifizierung}

\subsection{Lösungsorientierte Nutzer}
\subsection{Powernutzer}
\subsection{Gelegenheitsnutzer}
Kurze Erklärung der Klassifizierung und Herleitung der Ziele der einzelnen Nutzer
\label{ch:Content1:sec:Section2}
%% ===========================




%% content.tex
%%

%% ==============
\chapter{Lösungsansätze}

\section{Klassische Online-Hilfe}

\section{Lernspiel-App}

\section{Expertensystem}
\label{ch:Content2}
%% ==============



\chapter{Empfehlung}

\chapter{Zusammenfassung mit Fazit}
