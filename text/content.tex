%% content.tex
%%

\chapter{Status Quo}
In diesem Kapitel wollen wir den aktuellen Stand der Wissenschaft vorstellen. Hierbei beziehen wir uns auf das Thema Lernen und Online-Hilfe. Welches Wissen gibt es hierbei bereits und auf welchem Stand sind diese? Das wollen wir in diesem Kapitel klären, bevor es zu den Grundlagen dieser Arbeit kommt

\section{Lernen}
Ein Mensch lernt sein Leben lang. Sei es im Beruf, Privat oder bei seinen Hobbies. Die Schule hat das Ziel den Kindern bereits früh beizubringen, was richtiges Lernen ist. In der Theorie wollen wir zwei Möglichkeiten für das Lernen im Folgenden vorstellen.

\subsection{Problembasiertes Lernen}
">Problem based Lerning (PBL) is a learner-centred approach for designing learning environments in authentic, ill-structured problems"< entnommen aus \cite{zumbachproblembasiertes}. \par

Das Problembasierte Lernen, wie im Zitat deutlich wird, bezieht sich auf das Erarbeiten von selbstständigen Lösungsansätzen zum Lösen eines Problems. Der Lernende soll ein Thema selbstständig erarbeiten, ohne das jemand anderes einen Lösungsweg vorzeichnet. Auf dem Weg zu der Lösung eines Problems muss er evtl. mehrere Ansätze probieren um sein Ziel zu erreichen. Wie im Zitat bereits erwähnt steht der Lernende im Mittelpunkt. Der Lernende muss hierbei selbst die Initiative ergreifen. Ein Lehrer steht beim Lernprozess nur als Helfende Hand zur Seite. Er dient dabei lediglich als Coach, der evtl. aufkommende Fragen des Lernenden beantwortet und ihn in eine spezielle Richtung leitet, falls der Lernende nicht mehr weiterkommt. Beim Thema Software haben wir eine ähnliche Problemstellung. Ein möglicher Anwender kommt an einem Punkt in der Software nicht mehr weiter und muss selbstständig eine Lösung für das konkrete Problem finden. 
 
\subsection{Differenzielles Lernen}
Ein weitere Ansatz zum Thema Lernen gibt es in der Sporttheorie. Die Idee des differenziellen Lernens bezieht sich auf das Erlernen von Bewegungen im Sport. Beim differenziellen Lernen werden Bewegungen variiert. Beim Tischtennis beispielsweise wird ein Schlag auf verschiedene Wege durchgeführt. Hierbei werden vor allem Körperwinkel variiert. \footnote{Quelle: \cite{differenziellesLernen}} Es werden drei Bälle vom Lernenden gespielt und jede Bewegung sollte hierbei auf einem anderen Weg durchgeführt werden. Das Ziel dieser Form von Lernen ist es kein ideales Technikleitbild für eine Bewegung vorzugeben, also keine Lösung sondern die Bewegung so vielfältig wie möglich zu streuen. Wenn er dieses Training durchführt, dann kann ein Lernender in jeder Situation die korrekte Bewegung anwenden. Diese Form des Bewegungslernens ist sehr stark verwandt mit dem problembasierten Ansatz. Das Differenzielle Lernen gibt ebenfalls ein Problemstellung vor, die mit verschiedenen Lösungswegen gelöst werden soll. Der Unterschied hierbei besteht in der Vorgabe von Lösungsansätzen im Vergleich zum komplett selbständigen problematisierten Lernen.  Diese beiden Konzepte des Lernens wollen wir in dieser Arbeit für unsere Empfehlung verwenden.

\section{Online-Hilfe}


\chapter{Grundlagen}
\section{Gamification}

\section{Recommendersysteme}

%% ==============
\chapter{Adaption}
\label{ch:Content1}
%% ==============

\section{Adaption}
Klärung des Begriffes Adaption. Was bedeutet Adaption auf die Software bezogen.

\section{Adaption in unserer Arbeit}

Hinleitung der Adaption auf den Nutzer. Wie kommen wir zum Nutzer?
\dots




%% ===========================
\chapter{Nutzer}

\section{Nutzerrollen}
\subsection{Professoren}
\subsection{Studenten}
\subsection{Studierendensekretariat}
\subsection{Admins}

\section{Nutzer Klassifizierung}

\subsection{Lösungsorientierte Nutzer}
\subsection{Powernutzer}
\subsection{Gelegenheitsnutzer}
Kurze Erklärung der Klassifizierung und Herleitung der Ziele der einzelnen Nutzer
\label{ch:Content1:sec:Section2}
%% ===========================




%% content.tex
%%

%% ==============
\chapter{Lösungsansätze}

\section{Klassische Online-Hilfe}

\section{Lernspiel-App}

\section{Expertensystem}
\label{ch:Content2}
%% ==============



\chapter{Empfehlung}

\chapter{Zusammenfassung mit Fazit}


